\documentclass
[
%compress, notes=only
]
{beamer}
%\usetheme{Jane2}
\usetheme{Jane2}

\usecolortheme[rgb={0.2,0.3,0.75}]{structure}

%\usetheme{Jane1}
%\setkeys{Gin}{width=\linewidth, height=0.8\textheight, keepaspectratio=true}
%% LOCALIZATION
\usepackage[spanish]{babel}\decimalpoint % spanish hyphenization
\usepackage[latin1]{inputenc}

%% FONTS
\usepackage[mtbold]{mathtime}

%% GRAPHICS
\usepackage{graphicx,color}
\graphicspath{
   {/home/jane/MHDHII/RESUBMIT/PACKAGE/},
   {/home/jane/MHDHII/TALKMEX/},
   {/home/jane/IAU270/TALK/}
}

%% MOVIES
\usepackage{movie15}

%% DOCJUMP to get round too many videos problem
\usepackage{docjump}

%% tikz package for more graphics flexibility
\usepackage{tikz,pgf}

%% PERSONAL
\newcommand\red{\textcolor{red}}
\newcommand\green{\textcolor{green}}
\newcommand\darkgreen{\textcolor{green!85!black}}
\newcommand\blue{\textcolor{blue}}
\newcommand\darkblue{\textcolor{blue!85!black}}
\newcommand\white{\textcolor{white}}

%% FRONTMATTER
\title[3D MHD Simulations of Turbulent HII Regions] % (optional\, use only with long paper titles)
{\Large 3D MHD Simulations of HII Region Expansion in Turbulent Molecular Clouds}

\author{Jane Arthur}

\institute[CRyA-UNAM] % (optional, but mostly needed)
{
  Centro de Radioastronom\'{\i}a y Astrof\'{\i}sica\\
  Universidad Nacional Aut\'{o}noma de M\'exico\\
  Campus Morelia
}
\date[IAUNAM/INAOE] % (optional, should be abbreviation of conference name)
{23-25 marzo 2011, IAUNAM y INAOE}
%{\includegraphics{14sncyt.jpg}}
\subject{Astrophysics}

%\AtBeginSection[] % Do nothing for \section*
%{
%  \begin{frame}<beamer>
%    \frametitle{Esquema}
%    \tableofcontents[currentsection]
%  \end{frame}
%}

\defbeamertemplate{background canvas}{my bg}[1]{\includegraphics[height=\paperheight]{#1}}
\begin{document}
\setbeamertemplate{navigation symbols}{}
%\setbeamertemplate{blocks}{rounded}[shadow=false]

\newlength\movieheight
\setlength\movieheight{0.9\textheight}
\newlength\moviewidth
\setlength\moviewidth{1.1996\textheight}
\newlength\maxheight
\setlength\maxheight{0.8\textheight}


\newcommand\mymovie[1]{%
  \setkeys{Gin}{height=\movieheight}
  \includemovie
  [
  autoplay,
  autopause,
  repeat
%  controls,
%   text=\includegraphics{#1-0001.png},
  ]
  {\moviewidth}{\movieheight}{#1}
}

{
\setbeamercolor{background canvas}{bg=white!90!blue}
\begin{frame}[plain]
  \titlepage
%\centering
%\centerline{\includegraphics[width=\textwidth]{iau270}}
\end{frame}
}
{
\setbeamercolor{background canvas}{bg=black}
\begin{frame}[plain]
\centerline{\Large \white{The Team}}
\vspace*{0.03\textheight}
\begin{columns}
\column{0.33\linewidth}
\centerline{\white{Jane Arthur}}
\vspace*{0.005\textheight}

\centerline{\includegraphics[width=0.6\textwidth]{Jane_crop}}
\column{0.33\linewidth}
\centerline{\white{Will Henney}}
\vspace*{0.005\textheight}

\centerline{\includegraphics[width=0.6\textwidth]{Will_crop}}
\centerline{\white{Fabio De Colle}}
\vspace*{0.005\textheight}

\centerline{\includegraphics[width=0.6\textwidth]{Fabio_crop}}
\column{0.33\linewidth}
\centerline{\white{Garrelt Mellema}}
\vspace*{0.005\textheight}

\centerline{\includegraphics[width=0.6\textwidth]{Garrelt_crop}}
\centerline{\white{Enrique V\'azquez}}
\vspace*{0.005\textheight}

\centerline{\includegraphics[width=0.6\textwidth]{Enrique_crop}}
\end{columns}
\end{frame}
}
%%THE MONEY SHOT
{
\setbeamercolor{background canvas}{bg=black}
\begin{frame}[plain]
\centering
\centerline{\includegraphics[height=0.9\textheight]{movie-Ostar-stitchup-nolabels-019}}
%put movie in here
\end{frame}
}
%CODE DEVELOPMENT
%%(MAGNETO)HYDRODYNAMICS
%%RADIATIVE TRANSFER
%%HEATING AND COOLING
%%INITIAL CONDITIONS
{
\setbeamercolor{background canvas}{bg=white!90!blue}
\begin{frame}[plain]
\centerline{\large C\'ODIGO NUM\'ERICO}
\vfill
\centering
\begin{itemize}
\item<1-> Magnetohidrodin\'amica
\begin{itemize}
\item<2->\blue{MHD ideal}
\item<2->\blue{M\'etodo Godunov \textit{upwind} [De Colle \& Raga 2006]}
\item<2->\blue{$\nabla \cdot \mathbf{B} = 0$: transporte restringido
  [T\`oth 2000]}
\item<4->\darkgreen{\textit{Entropy Fix} [Balsara \& Spicer 1999]}
\item<4->\darkgreen{Condiciones de frontera}
\item<5->\red{Variables centradas}
\end{itemize}
\vfill
\item<1-> Transferencia Radiativa
\begin{itemize}
\item<3->\blue{M\'etodo C$^2$-Ray: [Mellema et al.\ (2006)]}
\end{itemize}
\vfill
\item<1-> Enfriamiento y Calentamiento
\vfill
\item<1-> Condiciones iniciales
\vfill
\item<1-> Paralelizado con Open-MP
\vfill
\end{itemize}
\end{frame}
}
{
\setbeamercolor{background canvas}{bg=white!90!blue}
\begin{frame}[plain]
\centerline{\large ENFRIAMIENTO Y CALENTAMIENTO EN EL GAS NEUTRO}
\vfill
\centering
\begin{itemize}
\item<1->Enfriamiento
 \begin{itemize}
\item<2->\blue{L\'{\i}neas excitadas por colisiones}
\item<2->\blue{de iones, \'atomos o mol\'eculas }
  \end{itemize}
\vfill
\item<1->Calentamiento
\begin{itemize}
     \item<3->\red{ absorpci\'on de luz estelar EUV por gas ionizado y polvo}
     \item<3->\red{ absorpci\'on de luz estelar FUV por gas neutro y polvo}
     \item<3->\red{ absorpci\'on de luz estelar rayos X por gas molecular y polvo }
  \end{itemize}
\end{itemize}
\vfill
\end{frame}
}
%\end{document}
{
\setbeamercolor{background canvas}{bg=white!90!blue}
\begin{frame}[plain]
\centerline{\large ENFRIAMIENTO EN EL GAS NEUTRO}
\vfill
\centering
\centerline{\includegraphics[width=0.95\textwidth]{plot-pdr-cool}}
\vfill
\centerline{Ajustes a modelos de Cloudy [Henney et al.\ 2009]}
\vfill
\end{frame}
}
{
\setbeamercolor{background canvas}{bg=white!90!blue}
\begin{frame}[plain]
\centerline{\large CALENTAMIENTO EN EL GAS NEUTRO}
\vfill
\centering
\centerline{\includegraphics[height=0.9\textheight]{nicmosorioncluster}}
\end{frame}
}
{
\setbeamercolor{background canvas}{bg=white!90!blue}
\begin{frame}[plain]
\centerline{\large CALENTAMIENTO EN EL GAS NEUTRO}
\vfill
\centering
\centerline{\includegraphics[width=0.45\textwidth]{fatuzzo_L_euv}~%
\includegraphics[width=0.45\textwidth]{fatuzzo_L_fuv}}
\vfill
\centerline{Distribuciones de flujos EUV y FUV para c\'umulos de
  estrellas}
\centerline{[Fatuzzo \& Adams 2008]}
\vfill
\end{frame}
}
{
\setbeamercolor{background canvas}{bg=white!90!blue}
\begin{frame}[plain]
\centerline{\large CALENTAMIENTO EN EL GAS NEUTRO}
\vfill
\centering
\centerline{\includegraphics[width=0.95\textwidth]{plot-pdr-heat}}
\vfill
\centerline{Ajustes a modelos de Cloudy [Henney et al.\ 2009]}
\vfill
\end{frame}
}
%TEST PROBLEMS
%%UNIFORM MEDIUM HD
%{
%\setbeamercolor{background canvas}{bg=green}
%\begin{frame}[plain]
%  \includegraphics[height=\textheight]{radii_vs_t_unif-weak-zerob-256}
%\end{frame}

%\begin{frame}[plain]
%  \includegraphics[height=\textheight]{velocities_vs_t_unif-weak-zerob-256}
%\end{frame}
%}
%%UNIFORM MEDIUM MHD
{
\setbeamercolor{background canvas}{bg=white!90!blue}
\begin{frame}[plain]
\centerline{PROBLEMAS DE PRUEBA}
\vfill
\centering
\begin{itemize}
\item<1->Expansi\'on de una regi\'on H\,{\sc ii} en un medio uniforme
\vfill
\item<1->Expansi\'on MHD de una regi\'on H\,{\sc ii} en un medio
  uniforme
\begin{itemize}
\item<-2>Par\'ametros de Krumholz et al.\ [2007]
\item<-2>$n_0 = 100$~cm$^{-3}$; $T_0 = 11$~K;\\ $B_0 = 14.2\,\mu$G;
  $Q_\mathrm{H} = 4\times10^{46}$~s$^{-1}$
\end{itemize}
\end{itemize}
\vfill
\end{frame}
}
{
\setbeamercolor{background canvas}{bg=black}
\begin{frame}[plain]
\centering
  \includegraphics[width=0.95\textwidth]{mhdcuts-B30krum-stitchup-nolabels-020}
%movie in here
\end{frame}
\begin{frame}[plain]
\centerline{\color{white}{Medio magn\'etico uniforme\hfill \(0\textrm{--}2~\mathrm{Myr}\)}}
\vfill
\includemovie[label=krum, autoplay, autopause, repeat]
{\textwidth}{0.69875\textwidth}{movies/mhdcuts-B30krum-stitchup-nolabels.avi}\\\small
\color{white!30!black}{Cold} \color{white!70!black}{neut}\color{white}{ral}
\(\bullet\)
\color{red!10!blue!50!white!50!black}{Warm} \color{red!10!blue!30!white}{neutral}
\(\bullet\)
\color{blue!80!black}{Parti}\color{green!80!black}{ally io}\color{yellow!80!black}{nized}
\(\bullet\)
\color{red!70!black}{Fully} \color{red!90!white}{ionized}
\end{frame}
\begin{frame}[plain]
\centering
  \includegraphics[width=0.95\textwidth]{mhdcuts-B30krumx-stitchup-nolabels-060}
%movie in here
\end{frame}
\begin{frame}[plain]
\centerline{\color{white}{Medio magn\'etico uniforme\hfill
  \(2\textrm{--}7~\mathrm{Myr}\)}}
\vfill
\includemovie[label=krumx, autoplay, autopause, repeat]
{\textwidth}{0.69875\textwidth}{movies/mhdcuts-B30krumx-stitchup-nolabels.avi}\\\small
\color{white!30!black}{Cold} \color{white!70!black}{neut}\color{white}{ral}
\(\bullet\)
\color{red!10!blue!50!white!50!black}{Warm} \color{red!10!blue!30!white}{neutral}
\(\bullet\)
\color{blue!80!black}{Parti}\color{green!80!black}{ally io}\color{yellow!80!black}{nized}
\(\bullet\)
\color{red!70!black}{Fully} \color{red!90!white}{ionized}
\end{frame}
}
%\begin{frame}[plain]
%  \includegraphics[height=\textheight]{radii_vs_t_krum}
%\end{frame}
%\begin{frame}[plain]
%  \includegraphics[height=\textheight]{velocities_vs_t_krum}
%\end{frame}

%HII REGION IN TURBULENT MAGNETIZED MEDIUM
%%MORPHOLOGIES AND IMAGES
{
\setbeamercolor{background canvas}{bg=white!90!blue}
\begin{frame}[plain]
\centerline{EXPANSI\'ON MHD DE UNA REGI\'ON H~II EN UN MEDIO TURBULENTO}
\vfill
\centerline{Condiciones iniciales: V\'azquez-Semadeni et al.\ [2005]}
\vfill
\begin{itemize}
    \item<1-> Simulaci\'on turbulenta MHD independiente de escala
  \begin{itemize}
  \item<2-> \blue{  $M_\mathrm{iso} = \sigma/c = 10$; \\$J = L/L_\mathrm{j} = 4$;\\
    $\beta = P_\mathrm{ter}/P_\mathrm{mag} = 0.1$ inicialmente}
  \end{itemize}
\item<1-> Fijamos la escala
\darkgreen{    \begin{itemize}
  \item<3-> $L = 4$~pc
    \item<3-> $T_0 = 5$~K; $\langle n_0 \rangle = 10^3$~cm$^{-3}$;
    $\langle B_0 \rangle = 14.2\,\mu$G
     \item<3-> $\beta = 0.032$; $B_\mathrm{rms} = 24.16\,\mu$G
       \item<3-> $n_\mathrm{max} > 10^6$~cm$^{-3}$  
  \end{itemize} }
\item<1-> Estrella 
\red{   \begin{itemize}
   \item<4-> $T_\mathrm{eff} = 37\,500$~K; $M_* = 32\,M_\odot$
      \item<4-> `tipo O9': $Q_\mathrm{H} = 10^{48.5}$~s$^{-1}$
      \item<4-> `tipo B0.5': $Q_\mathrm{H} = 5\times10^{46}$~s$^{-1}$ 
  \end{itemize} }
\end{itemize}
\vfill
\end{frame}
}
{
\setbeamercolor{background canvas}{bg=black}
\begin{frame}[plain]
\centerline{\includegraphics[height=0.9\textheight]{movie-Ostar-stitchup-nolabels-019}}
\end{frame}
\begin{frame}[plain]
\centerline{\color{white}{Estrella O en un medio turbulento}}
\vfill
  \begin{columns}
    \column{0.7\linewidth}
    \includemovie[label=ostar, autoplay, autopause, repeat, controls]
    {0.929\maxheight}{\maxheight}{movies/movie-Ostar-stitchup-nolabels.avi}
    \column{0.3\linewidth} 
    \begin{block}{Arriba}
      \color{red}{[N\,II]}\quad
      \color{green}{H$\alpha$}\quad
      \color{blue}{[O\,III]}
    \end{block}
    \begin{block}{Abajo}
      \color{red}{FIR polvo fr\'{\i}o}\\
      \color{green}{MIR PAHs tibias}\\
      \color{blue}{Radio Libre-libre}
    \end{block}
  \end{columns}
\end{frame}
\begin{frame}[plain]
\centerline{\includegraphics[height=0.9\textheight]{movie-Bstar-stitchup-nolabels-100}}
\end{frame}
\begin{frame}
\centerline{\color{white}{Estrella B en un medio turbulento}}
\vfill
  \begin{columns}
    \column{0.7\linewidth}
    \includemovie[label=bstar, autoplay, autopause, repeat, controls]
    {0.929\maxheight}{\maxheight}{movies/movie-Bstar-stitchup-nolabels.avi}
    \column{0.3\linewidth} 
    \begin{block}{Upper panels}
      \color{red}{[S\,II]}\quad
      \color{green}{[N\,II]}\quad
      \color{blue}{H$\alpha$}
    \end{block}
    \begin{block}{Abajo}
      \color{red}{FIR polvo fr\'{\i}o}\\
      \color{green}{MIR PAHs tibias}\\
      \color{blue}{Radio Libre=libre}
    \end{block}
  \end{columns}
\end{frame}

}
{
\setbeamercolor{background canvas}{bg=white!90!blue}
\begin{frame}[plain]
\centerline{\large MORFOLOG\'{I}AS e IM\'AGENES}
\vfill
\begin{itemize}
\item<1-> Estrella O
 \begin{itemize}
\item<2-> \blue{  Estructura irregular con filamentos (dedos)}
    \item<2->\blue{   Campo magn\'etico no afecta morfolog\'{\i}a global}
\item<2->\blue{   CM afecta estructuras a peque\~na escala}
\item<2-> \blue{  CM soporta filamentos contra la implosi\'on radiativa }
\end{itemize}
\item<1-> Estrella B
\begin{itemize}
\item<3-> \darkgreen{  Regi\'on HII m\'as esf\'erica que para estrella O, y sin
  filamentos}
\item<3-> \darkgreen{  Regiones de alta densidad est\'an evaporadas en la PDR}
\end{itemize}
\item<4-> \red{ Diferencia entre estrellas O y B es mayor que entre HD y MHD}
\end{itemize}
\vfill
\end{frame}
}
{
\setbeamercolor{background canvas}{bg=black}
\begin{frame}[plain]
\centerline{\large \color{white}{COMPARACI\'ON CON RCW\,120}}
\vfill
\centering
\centerline{\includegraphics[width=0.45\textwidth]{RCW120}~\hfill%
\includegraphics[width=0.45\textwidth]{RCW120b}}
\end{frame}
}
{
\setbeamercolor{background canvas}{bg=white!90!blue}
\begin{frame}[plain]
\centerline{\large{COMPARACI\'ON CON RCW\,120}}
\vfill
\begin{itemize}
\item $D \sim 3.5$~pc a 1.35~kpc 
\item Estrella \'unica ionizadora, tipo O6-O8
\item $T_\mathrm{eff} = 37.5\pm2$~kK
\item $Q_\mathrm{H} = 48.58\pm0.22$
\item $M_\mathrm{870} \sim 1100$--$2100\,M_\odot$
\item Edad $< 5$~Myr
\item \red{Concluimos $\mathrm{Edad} \sim 200\,000$~yrs}
\end{itemize}
\vfill
\centerline{{\small{Zavagno et al.\ [2007,2010]; Deharveng et al.\
    [2009]}}}
\centerline{{\small{Anderson et al.\ [2010]; Martins et al.\ [2010] }}}
\end{frame}
}
%\end{document}

%%GLOBAL PROPERTIES
{
\setbeamercolor{background canvas}{bg=white!90!blue}
%\begin{frame}[plain]
%\includegraphics[height=\textheight]{comparison1_vs_t_Ostar}
%\end{frame}
%\begin{frame}[plain]
%\includegraphics[height=\textheight]{comparison1_vs_t_Bstar}
%\end{frame}
%\begin{frame}[plain]
%\includegraphics[height=\textheight]{comparison2_vs_t_Ostar}
%\end{frame}
%\begin{frame}[plain]
%\includegraphics[height=\textheight]{comparison2_vs_t_Bstar}
%\end{frame}
\begin{frame}[plain]
\centerline{Expansi\'on}
\vfill
\centering
\centerline{\includegraphics[width=0.45\textwidth]{comparison2b_vs_t_Ostar}~%
\includegraphics[width=0.45\textwidth]{comparison2b_vs_t_Bstar}}
\vfill
\centerline{\hfill Estrella O \hfill Estrella B \hfill}
\vfill
\centerline{S\'{\i}mbolos: MHD}
\end{frame}
\begin{frame}[plain]
\centerline{Velocidades del gas}
\vfill
\centering
\centerline{\includegraphics[height=0.8\textheight]{comparison3_vs_t_Ostar}~%
\includegraphics[height=0.8\textheight]{comparison3_vs_t_Bstar}}
\vfill
\centerline{\hfill Estrella O \hfill Estrella B \hfill}
\end{frame}
}
%%MAGNETIC QUANTITIES
{
\setbeamercolor{background canvas}{bg=white!90!blue}
%\begin{frame}[plain]
%\includegraphics[height=\textheight]{mhdcuts-Bstar-stitchup-nolabels-100}
%\end{frame}
\begin{frame}[plain]
\centerline{CANTIDADES MAGN\'ETICAS}
\vfill
\centering
\centerline{\includegraphics[width=0.45\textwidth]{mhd-pressures-rgb-Bstar-ep-1000-n-B}~\hfill~%
\includegraphics[width=0.4\textwidth]{HS5}}
\vfill
\centerline{Resultados num\'ericos \hfill Harvey-Smith et al
  [en prep]}
\end{frame}
}
%\begin{frame}[plain]
%\includegraphics[height=\textheight]{mhd-pressures-rgb-Bstar-ep-1000-pram-pmag}
%\end{frame}

%DISCUSSION
%%PROJECTED B FIELD
{
\setbeamercolor{background canvas}{bg=white!90!blue}
\begin{frame}[plain]
\centerline{Campo Magn\'etico Proyectado: T\'ecnicas Observacionales}
\vfill
\begin{itemize}
\item<1-> Componente paralela a la l\'{\i}nea de visi\'on
\begin{itemize}
\item<2-> \blue{Medida de rotaci\'on de Faraday (gas ionizado)}
\item<2-> \blue{Espectroscop\'{\i}a Zeeman (gas neutro o molecular)}
\end{itemize}
\item<1-> Componente en el plano del cielo
\begin{itemize}
\item<3-> \blue{Emisi\'on/absorpci\'on polarizada de granos alineados}
\item<3-> \blue{M\'etodo Chandrasekhar-Fermi (t\'ecnica estad\'{\i}stica)}
\end{itemize}
\item<4-> Resultados num\'ericos: par\'ametros de Stokes
\end{itemize}
\end{frame}
}
{
\setbeamercolor{background canvas}{bg=black}
\begin{frame}[plain]
\centerline{\color{white}{Campo Magn\'etico Proyectado: Estrella O}}
\vfill
\centerline{\includegraphics[height=0.7\textheight]{bproj-Ostar-t200-full}}
\vfill
\centerline{\color{white}{Grises: l\'{\i}nea de visi\'on; Vectores rojos: plano del cielo}}
\end{frame}
\begin{frame}[plain]
\centerline{\color{white}{Campo Magn\'etico Proyectado: Estrella B}}
\vfill
  \centerline{\includegraphics[height=0.7\textheight]{bproj-Bstar-t1000-full}}
\vfill
\centerline{\color{white}{Grises: l\'{\i}nea de visi\'on; Vectores rojos: plano del cielo}}
\end{frame}
\begin{frame}[plain]
\centerline{\color{white}{Enfoque en gl\'obulo denso: Estrella O}}
\vfill
  \centerline{\includegraphics[height=0.7\textheight]{bproj-Ostar-t200-globule-south}}
\vfill
\end{frame}
\begin{frame}[plain]
\centerline{\color{white}{Enfoque en gl\'obulo denso: Estrella B}}
\vfill
\centerline{\includegraphics[height=0.7\textheight]{bproj-Bstar-t1000-globule-south}}
\vfill
\end{frame}
}
{
\setbeamercolor{background canvas}{bg=white!90!blue}
\begin{frame}[plain]
\centerline{Campo Magn\'etico Proyectado: Resumen}
\vfill
\begin{itemize}
\item<1-> Orden a gran escala en el gas neutro
\vfill
\item<2-> En los gl\'obulos densos: Estrella O
 \begin{itemize}
    \item<2->\blue{Doblado en el gas molecular}
    \item<2-> \blue{Perpendicular al frente de ionizaci\'on en el gas ionizado}
\end{itemize}
\vfill
\item<3-> En los gl\'obulos densos: Estrella B
\begin{itemize}
    \item<3-> \blue{Doblado en el gas molecular}
    \item<3-> \blue{En el gas neutro/ionizado: dominado por material difuso en la
    l\'{\i}nea de visi\'on.}
  \end{itemize}
\vfill
\item<4-> Gl\'obulos no parecen a los casos idealizados [Henney et al.\ 2009]
\end{itemize}
\vfill
\end{frame}
\begin{frame}[plain]
\centerline{CONCLUSIONES GENERALES}
\vfill
\begin{itemize}
\item En las escales de tiempo de las simulaciones, el campo
  magn\'etico no afecta la expansi\'on de la regi\'on HII
\item Hay m\'as fragmentaci\'on y gl\'obulos m\'as densos en el caso
  de la estrella O
\item Radiaci\'on FUV no-ionizante es importante en el caso de la
  estrella B.
\item Expansi\'on de la regi\'on HII y PDR borran estructura fina
  desordenada en el campo magn\'etico y producen orden a gran escala
  en el cascar\'on neutro.
\item Gl\'obulos densos tienen campos longitudinales que se doblan en
  la cabeza, en el gas neutro/molecular.
\item Dispersi\'on de velocidades en el gas ionizado es de orden
  8~km~s$^{-1}$: inicialmente por el flujo de champ\'an y luego por
  los flujos fotoevaporados.
\end{itemize}
\vfill
\end{frame}
\begin{frame}[plain]
\centerline{Todavia por hacer\dots}
\vfill
\begin{itemize}
\item Presi\'on de radiaci\'on
\vfill
\item Vientos estelares
\vfill
\item Autogravidad
\vfill
\item Fuentes ionizadoras multiples
\vfill
\item etc
\end{itemize}
\vfill
\end{frame}
\begin{frame}[plain]
\centerline{PROXIMAMENTE}
\vfill
\centerline{\includegraphics[width=0.8\textwidth]{mnr_18507_LR}}
\vfill
\centerline{Vis\'{\i}tenos en YouTube: \texttt{www.youtube.com/divBequals0}}
\vfill
\end{frame}
}
\end{document}
