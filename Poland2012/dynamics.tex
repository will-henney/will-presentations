% Created 2012-07-11 Wed 15:29
\documentclass[presentation]{beamer}
\usepackage[utf8]{inputenc}
\usepackage[T1]{fontenc}
\usepackage{fixltx2e}
\usepackage{graphicx}
\usepackage{longtable}
\usepackage{float}
\usepackage{wrapfig}
\usepackage{soul}
\usepackage{textcomp}
\usepackage{marvosym}
\usepackage{wasysym}
\usepackage{latexsym}
\usepackage{amssymb}
\usepackage{hyperref}
\tolerance=1000
\providecommand{\alert}[1]{\textbf{#1}}

\title{Dynamics of the Orion Nebula}
\author{William Henney}
\date{2012-07-01}
\hypersetup{
  pdfkeywords={},
  pdfsubject={},
  pdfcreator={Emacs Org-mode version 7.8.06}}

\usepackage{etex} \usepackage{will-beamer-poland}
\begin{document}

\maketitle

\begin{frame}
\frametitle{Outline}
\setcounter{tocdepth}{3}
\tableofcontents
\end{frame}





\section{Preamble}
\label{sec-1}
\begin{frame}
\frametitle{Abstract}
\label{sec-1-1}


I review what we know about the dynamics of photoionized regions in general and the Orion Nebula in particular.  I will cover the fundamental physical processes that are important in setting ionized gas in motion: thermal pressure gradients, stellar winds, jets and other outflows, radiation pressure, magnetic fields, instabilities, turbulence.  I will also describe results from recent numerical simulations and observations of the kinematics of the ionized gas in Orion.  Some questions that I will address are:  Why does Orion not look like a Strömgren sphere?  What is the cause of all of the structure that we see?  Does the stellar wind have any effect on the dynamics of the nebula?  When is it OK to just ignore the dynamics?
\end{frame}
\section{Ingredients}
\label{sec-2}
\begin{frame}
\frametitle{What drives nebular dynamics?}
\label{sec-2-1}


\begin{itemize}
\item Inviscid momentum equation
\end{itemize}
\[
\frac{d}{dt} ( \rho \VEC{u} ) + \VEC{\grad} ( P + \rho u^2 ) = \rho \VEC{a} 
\]
\end{frame}
\begin{frame}
\frametitle{Font test}
\label{sec-2-2}


\[
abcdefghijklmnopqrstuvwxyz
\]

\ion{O}{3} or O\,\textsc{iii}

\[
x^2 + y^2 = z^2 \sin^2 \alpha
\]
\end{frame}

\end{document}
