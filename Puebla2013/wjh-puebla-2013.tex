\documentclass[presentation]{beamer}
\usepackage[utf8]{inputenc}
\usepackage[T1]{fontenc}
\usepackage{fixltx2e}
\usepackage{textcomp}
\usepackage{hyperref}
\usepackage{siunitx}
\usepackage{booktabs}
\usepackage{media9}
\usepackage{etex} 
\usepackage{pgfpages}
\usepackage{tikz}
\usepackage{will-beamer-puebla} 
\usepackage{xcolor}
\tolerance=1000

\graphicspath{ 
  {figs/},
  {../Baltimore2013/figs/},
  {../TownsvillePoster/figs/},
  {../LarimTalk2010/movies/},   
}

\newcommand\backskip{\vspace*{-\baselineskip}}
\newcommand\backsmallskip{\vspace*{-\smallskipamount}}
\newcommand\backmedskip{\vspace*{-\medskipamount}}
\newcommand\backbigskip{\vspace*{-\bigskipamount}}

\setkeys{Gin}{width=\linewidth, height=0.8\textheight, keepaspectratio=true}

% \setbeameroption{show notes}

% \renewcommand\baselinestretch{1.2}

\title{Dynamics of ionized gas around massive young star clusters}
\author{\textit{William J. Henney}}
\date[Baltimore 2013]{December 2013 \(\cdot\) Puebla, Mexico}
\institute[CRyA, UNAM]
{
  \structure{Centro de Radioastronomía y Astrofísica\\
  UNAM, Morelia, México}
}

\hypersetup{
  pdfkeywords={Orion Nebula, Astrophysics, Dynamics},
  pdfsubject={},
  pdfcreator={Lovingly hand-crafted by the author using pdflatex and beamer}
}


\AtBeginSection[]
{
  \begin{frame}<beamer>
    \frametitle{Coming up next \dots}
    \tableofcontents[
    sectionstyle=show/shaded,
    currentsubsection, 
    hideothersubsections
    ]
  \end{frame}
}



\begin{document}

\maketitle

\begin{frame}
\frametitle{Principal collaborators}

\begin{block}{CRyA-UNAM, Morelia, Mexico}
\begin{description}
\item[\small HD] \textit{Jane Arthur}
\item[\small Turbulence] \textit{Enrique Vázquez-Semadeni}
\item[\small Students] \textit{Sac-Nicté Serrano Medina}
\item \textit{A. Jorge Tarango Yong}
\item \textit{Luis Ángel Gutierrez Soto} 
\end{description}
\end{block}

\begin{block}{Elsewhere}
  \begin{description}
  \item[\small MHD] \textit{Fabio de Colle} (ICN-UNAM, Mexico)
  \item[\small Radiation] \textit{Garrelt Mellema} (Stockholm Observatory, Sweden)
  \item[\small Observations] \textit{María Teresa García-Díaz} (IA-UNAM, Ensenada, Mexico)
  \item \textit{Bob O'Dell} (Vanderbilt, USA)
  \item \textit{Bob Rubin} \textbf{[\textdagger{} 2013-03-03]} (NASA-AMES/Orion Enterprises)
  \end{description}
\end{block}

\end{frame}

\begin{frame}<beamer>
  \frametitle{Dynamics of the Orion Nebula}
  \tableofcontents[hidesubsections]
\end{frame}

\section{Whatever}
\subsection{Turbulent models}

\newlength\maxheight
\setlength\maxheight{0.8\textheight}
\newlength\moviewidth
\setlength\moviewidth{0.7\textwidth}

\begin{frame}
\frametitle{Turbulent models: initial conditions}
\includegraphics{poland-figs/rgb-CPF-initial}
\end{frame}

\begin{frame}[shrink=5]
\frametitle{Turbulent models: state of play}
\begin{block}{Physics we have}
  \begin{itemize}
  \item 3D time-dependent, hydrodynamics
  \item Approximate radiative transfer
  \item Microphysics:
    \begin{itemize}
    \item good for ionized gas
    \item fair for PDR
    \item poor for molecular gas
    \end{itemize}
  \item {}[Ideal magnetohydrodynamics]
  \end{itemize}
\end{block}
\begin{block}{Physics we lack}
  \begin{itemize}
  \item Stellar winds
  \item Radiation pressure
  \item Diffuse field
  \item Self-gravity
  \item {\footnotesize Better microphysics, better radiative transfer,
    \scriptsize multifluids, non-ideal MHD, \tiny \(\upkappa\)-distributions, etc \dots}
  \end{itemize}
\end{block}
\end{frame}

\begin{frame}[plain]%%
  \newlength\figwidth
  \setlength\figwidth{0.33\textwidth}
  \renewcommand\arraystretch{0.0}
  \setlength\tabcolsep{0pt}
  \graphicspath{
    {poland-figs/movie-stills/O-Star-512-PDR-2012/},
    }
  \begin{tabular}{lll}
    \includegraphics[width=\figwidth]{01-Opening-Titles}
    & 
    \includegraphics[width=\figwidth]{02-Model-Parameters}
    & 
    \includegraphics[width=\figwidth]{03-Color-Scheme}
    % & 
    % \includegraphics[width=\figwidth]{04-Evolution-Start}
    \\
    \includegraphics[width=\figwidth]{05-Evolution-Mid}
    & 
    \includegraphics[width=\figwidth]{06-Evolution-End}
    & 
    \includegraphics[width=\figwidth]{07-Detail-View}
    % & 
    % \includegraphics[width=\figwidth]{08-Swimming-Sisters} 
    \\
    % \includegraphics[width=\figwidth]{09-Long-Wav-Color-Scheme}
    % & 
    \includegraphics[width=\figwidth]{10-Long-Wav-Mid}
    & 
    \includegraphics[width=\figwidth]{11-Long-Wav-End}
    & 
    \includegraphics[width=\figwidth]{12-Simulations-Credit}
\end{tabular}

\end{frame}


\begin{frame}
  \frametitle{Turbulent models: results}
  \begin{columns}
    \column{0.6\linewidth}
    \begin{itemize}
    \item Many morphological features of observed \hii{} regions are
      reproduced naturally
      \begin{itemize}
      \item Due to existing density structure in the
        turbulent molecular cloud, combined with fragmentation induced
        by interaction with the ionized gas
      \end{itemize}
    \item Velocity dispersions of order the sound speed are
      maintained in the ionized gas during the entire evolution
    \item The highest pressure neutral/molecular gas is driven to
      equipartition between thermal, magnetic, and turbulent energies
    \item Lower pressure gas bifurcates into zones dominated by one or
      the other
    \end{itemize}
    \column{0.4\linewidth}
    \includegraphics{poland-figs/comparison3_vs_t_Ostar}%
  \end{columns}
\end{frame}

\begin{frame}
  \frametitle{Turbulent models: more results}
  \begin{columns}
    \column{0.5\linewidth}
    \includegraphics{poland-figs/mhd-pressures-rgb-Ostar-et-0200-pram-pmag}
    \column{0.5\linewidth}
    \includegraphics{poland-figs/mhd-pressures-rgb-Ostar-et-0200-n-B}
  \end{columns}
\end{frame}

% {
%   \usebackgroundtemplate{
%     \parbox[c][\paperheight][c]{\paperwidth}{\missingmovietext}
%   }
\def\MovieFile{poland-figs/O-Star-512-PDR-2012.mov}
\def\MoviePlayer{StrobeMediaPlayback.swf} % or VPlayer.swf
\begin{frame}[plain]%%
\hspace*{-1ex}%
% \includemovie[label=new-movie, autoplay, autopause, controls, repeat=1]
% {1.27968\paperheight}{0.96\paperheight}{poland-figs/O-Star-512-PDR-2012.mov}
\includemedia[
  width=1.27968\paperheight,
  height=0.96\paperheight,
  activate=pageopen,
  addresource=\MovieFile,
  flashvars={src=\MovieFile&scaleMode=letterbox}
]{}{\MoviePlayer}
\end{frame}
% }



\end{document}
